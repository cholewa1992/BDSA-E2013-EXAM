\chapter{Proposed system}
\label{sec:proposed system}

\section{Overview}


\section{Functional requirements}
\begin{itemize}
	\setlength{\itemsep}{-5pt}
	\item Item example
\end{itemize}

\section{Nonfunctional requirements}

\emph{Easy to handle.} Or some other stuff (I'm writing these to have templates)

/* This section we did not use in our RAD. I don't know why it's here
\subsection{Usability}
text

\subsection{Reliability}

\subsection{Performance}

\subsection{Supportability}

\subsection{Implementation}

\subsection{Interface}

\subsection{Packaging}

\subsection{Legal}
*/

\section{System models}

\subsection{Scenarios}
In order to specify the use cases in which the user can interact with the program, it is required to initially specify a couple of different scenarios the user can have while utilizing the application.

The following scenarios were set up in order to analyze them:

\begin{itemize}
	\setlength{\itemsep}{-5pt}
	
	\item A user wants to...
	\item Another user wants to...
\end{itemize}

\subsection{Use case model}


/* This is an example. Should be replaced with an actual use cases */
\begin{itemize}
	\setlength{\itemsep}{-5pt}
	\item User
	\item Device
\end{itemize}

Through the inspection of the scenarios depicted in the Scenario’s section, we have deduced the use cases that the program should be able to support.

The use cases are as follows:

\begin{itemize}
	\setlength{\itemsep}{-5pt}
	
	\item Example 1
	\item Example 2
\end{itemize}

After refining the system design, we have added the following boundary scenarios:
\begin{itemize}
	\setlength{\itemsep}{-5pt}
	
	\item Start Program
	\item Shutdown Program
	\item Save Exception
	\item Unexpected Shutdown Exception
\end{itemize}

To have a more thorough understanding of each individual use cases, the following explanations was produced, describing each and every step of each use case.
The explanations furthermore include the relation between the actors and the use cases

\begin{center}
	\begin{tabular}{ | l | p{10cm} |  }
		 \hline
		Use Case Name & Make Event \\ \hline
		Participating Actors & Initiated by \emph{User} \\ \hline
		Flow of Events & \begin{enumerate}
						\item[1.] The \emph{user} requests a specific date
						\begin{enumerate}
							\item[2.] The system presents the \emph{user} with an inspection window for the date
						\end{enumerate}
						\item[3.] The \emph{user} requests the creation of a new event
						\begin{enumerate}
							\item[4.] The system presents the \emph{user} with an event window to input information about the event           to be created
						\end{enumerate}
						\item[5.] The \emph{user} inputs information about the new event and continues
						\begin{enumerate}
							\item[6.] The system prompts the \emph{user} to associate the events with a calendar
							\item[7.] The system provides the \emph{user} with a list of calendars to choose from. If not calendar is picked, the event is enlisted under the default calendar.
						\end{enumerate}
						\item[8.] The \emph{user} confirms the creation of the event
						\begin{enumerate}
							\item[9.] The system updates the application with the new event
						\end{enumerate}
					\end{enumerate} \\ \hline
		Entry Condition & \begin{itemize}
						\item The \emph{user} has the application open
					\end{itemize} \\ \hline
		Exit Condition & \begin{itemize}
						\item The \emph{user} successfully created the event
					\end{itemize} \\
		\hline
	\end{tabular}
\end{center}

\subsection{Defining entity, boundary and control objects}

\subsubsection{Entity objects}

\begin{enumerate}
	\item[1.] Calendar \hfill \\
	A calendar houses a number of events. It has a name and can be part of a 'tag', grouping together calendars
	\item[2.] Event \hfill \\
	An event is the backbone of every calendar. It has a name, description and a date, and must be placed in a specific calendar. Users can add reminders to an event.
	\item[3.] Tag \hfill \\
	A tag is coupled to one or more calendars. By 'tagging' calendars, a user can easily group multiple calendars together.
	\item[4.] Reminder \hfill \\
	A reminder is created along with, or later added to an event. It can be of several types (vibration, email, sound), and will notify the user at a specified time.
\end{enumerate}

\subsubsection{Boundary objects}

\begin{enumerate}
	\item[1.] Device \hfill \\
	The 'device', most likely a smartphone, is our only boundary device, as it houses everything we need to communicate with the user.
\end{enumerate}

\subsubsection{Control objects}

These controllers are all more or less self-explanatory. They enable the user to realise all use cases from their device.

\begin{enumerate}
	\item[1.] MakeEventController
	\item[2.] InspectEventController
	\item[3.] EditEventController 
	\item[4.] DeleteEventController
	\item[5.] InspectCalendarController
	\item[6.] MakeCalendarController
	\item[7.] DeleteCalendarController
	\item[8.] EditCalendarController 
	\item[9.] NotifyUserController 
\end{enumerate}

\subsection{Object model}



\subsection{Dynamic model}



\subsubsection{Sequence diagrams}



\subsubsection{State machines}



\newpage
