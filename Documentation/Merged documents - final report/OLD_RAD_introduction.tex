 \section{Division of labour}
              \begin{center}
                  \begin{tabular}{ | l | l | p{6cm} | p{4cm} |}
                  \hline
                  Task & owner \\ \hline
                  Database & jbec, mkin \\ \hline
                  Storage Subsystem & jbec \\ \hline
                  In memory Storage Subsystem & jbec \\ \hline
                  Entity framework Storage Subsystem & jbec \\ \hline
                  Web server subsystem & jkas, morr \\ \hline
                  Communication framework & mkin, mjup \\ \hline
                  My Movie API & jbec, jkas \\ \hline
                  WPF Client & mkin \\ \hline
                  ASP.net & jbec, mjup,  \\ \hline
                  Testing & jbec, mjup, jkas \\ \hline
                  Optimizing reponse time & jbec, mjup, jkas\\ \hline
                  RAD and SDD & all \\ \hline 
                  RAD refinement & morr, mifr \\ \hline 
                  Drawing diagrams & jbec, mifr, morr \\ \hline
                  Scrum documentation & mkin \\ \hline
                  Testing strategy & jkas \\ \hline
                  Product owner & mifr \\ \hline
                  Scrum master & mkin \\ \hline
                  \end{tabular}
              \end{center}
              
              Who have written each specific section of the documentation can be found in the revision histories\\
              \\
              All code is marked by a author tag showing who wrote the code
              

\section{Old RAD}
This is the previous version of the RAD, as it looked before we began to work on the code. It has since been reworked, so we chose to keep this artefact as part of the appendix.

\section{Old RAD - Introduction}
\label{sec:introduction}

\subsection{Revision History}
\begin{center}
    \begin{tabular}{ | l | l | p{6cm} | p{4cm} |}
    \hline
    Version & Date & Description & Author(s) \\ \hline
    0.1.0 & 19/11/2013 & Initial Document & MIFR, JKAS, MJUP, JBEC, MKIN, MORR
    \\ \hline
    0.2.0 & 20/11/2013 & Added scenarios and use cases & MIFR, JKAS, MJUP, JBEC, MKIN
    \\ \hline
    0.3.0 & 20/11/2013 & Added glossary, functional and non-functional requirements, object model, Use case diagram, State machine diagram and refined introduction formulation & MIFR, JKAS, MJUP, JBEC, MKIN
    \\ \hline
    0.4.0 & 21/11/2013 & Refined scenarios, added sequence diagrams, added boundary objects & MIFR, JKAS, MJUP, JBEC, MKIN, MORR
    \\ \hline
    \end{tabular}
\end{center}

\subsection{Purpose of the system}

The purpose of the system is to create a tool for people watching movies. We want to give our users access to a movie database available at their fingertips for inspiration and the possibility to read about their favourite movies.

Two clients shall be available . A browser based client enabling access from mobile devices as well as a desktop application for their personal computer.  


%Init formulation The purpose of the system is to create a tool for easy access to a movie database through the use of two clients: A desktop- and browser-version.


\subsection{Scope of the system}

The scope of the system is as follows

Scope is making a system containing most movies with related information and make this data available though a distributed RESTful web service. Two clients will also be implemented. An ASP.net web browser client / homepage and a desktop application done with WPF.

Part of the scope will also be to work with responsiveness enabling caching and database optimizing through indexing and optimized SQL quires.

Out of scope is looking into user security issues

%Init definition
%\begin{itemize}
%\item A Database which contains most movies related information
%\item An Application Server which exposes services making the Database %information accessible
%\item A Desktop Client which presents the user with movie information from the %Database
%\item A Web Client which also presents the user with movie information from the %Database
%\end{itemize} 

\subsection{Objectives and success criteria of the project}


The success criteria for this project is reached if end-users, in an easy and fast manner, are able to find movie information through both the desktop- and web-client. Success will be measured by how many users, without complications, are able to use our system to find desired information.

This goal is achieved if 9/10 users are able to do so.

%Init definition
%The success criteria for this project is reached if end-users, in an easy and fast manner, are able to find movie information through both the desktop- and web-client.

