\section{Proposed system}
\label{sec:proposed system}

\subsection{Overview}
A distributed system where a user is able to search, find and edit movies and persons. Information for the selected movie or person is displayed.
\subsection{Functional requirements}
\begin{itemize}
	\setlength{\itemsep}{-5pt}
	\item The following components shall be implemented
	\begin{itemize}
		\setlength{\itemsep}{-5pt}
		\item A database
		\item A web server exposing functions for handling data to and from the database
		\item A Windows Application
		\item A Web Client
	\end{itemize}
	\item All above parts shall be implemented in C\#. The Windows client in WPF, in .NET 4.5, and the web client in ASP.NET
	\item Communication between clients and server shall happen in a RESTful manner sending JSON over HTTP
	\item The server must build on the HttpListener for HTTP communication
	\item The server should support two ways of retrieving data. One being from its memory cache and the other being from the Database. If the requested data is in memory it will be retrieved from there instead of from the Database
	\item The server must be able to merge information from My Movie API. If a search term is not found in the database, the server will search IMDB using My Movie API. If there is a search result from the My Movie API, it will be sent to the client.
	\item The clients in the system should display information coming from the server, and should also take input from the user and transfer it back to the server.
\end{itemize}

\subsection{Nonfunctional requirements}

\begin{itemize}

\item[\textbf{Usability}]
\begin{itemize}
\item A user able to use an internet browser must be able to use the System.
\end{itemize}

\vspace{0.2cm}
\item[\textbf{Reliability}]
\begin{itemize}
\item Reliability is a \textit{must} for the system. The user should experience as few software crashes as possible, with a maximum of 2 per day.
\item The client application should function in a non-connected mode.
\item The server should work also if the database is unavailable.
\item Restarting the system is acceptable if a software crash does happen.
\item The system must not lose data as result of a software crash.
\end{itemize}


\vspace{0.2cm}
\item[\textbf{Performance}]

\hspace{50cm}

\begin{itemize}
\item The system should be optimized for fast response time by using caching and lazy loading.
\item System requirements
\end{itemize}


\begin{tabular}{|p{2cm}|p{5cm}|p{5cm}|}
\hline  & Desktop Client & Web Client \\ 
\hline Operation System & Windows 7/8 & Webkit2 browser \\
\hline Processor & Intel Core2 Duo T7500 or equivalent & 1.4 GHz Scorpion or equivalent \\ 
\hline Memory & 2gb RAM & 512mb RAM \\ 
\hline Storage & 200mb available & non-applicable \\ 
\hline 
\end{tabular} 

\vspace{0.2cm}

\begin{itemize}
\item The system should support at least 7 concurrent users
\item The server must support storing at least 10 gigabytes of data
\item The maximum latency the user can experience, when his internet is not a bottleneck, must not surpass 5 seconds
\end{itemize}

\vspace{0.2cm}
\item[\textbf{Supportability}]
\begin{itemize}
\item Extensibility
\begin{enumerate}
\item The system should be prepared for implementing more functionality which the logged in user can utilize
\item The system should be prepared for changing storage architecture
\end{enumerate}
\end{itemize}

\vspace{0.2cm}
\item[\textbf{Implementation}]
\begin{itemize}
\item The desktop client will be implemented for the Windows 7/8 operation system
\item The web client will be implemented with ASP.NET MVC
\end{itemize}

\vspace{0.2cm}
\item[\textbf{Interface}]
\begin{itemize}
\item The system shall interact with the My Movie API
\item Initial data for the movie database is made available as a .mdf file
\end{itemize}

\vspace{0.2cm}
\item[\textbf{Packaging}]
\begin{itemize}
\item The back-end server software and database is installed by us
\item The Clients are installed by the end-user
\item The desktop client should be portable and not require any installation
\end{itemize}

\end{itemize}


\subsection{System models}

\subsubsection{Scenarios}
In order to specify the use cases in which the user can interact with the program, it is required to initially specify a couple of different scenarios the user can have while utilizing the application.

The following scenarios were set up in order to analyze them:

\begin{center}
	\begin{tabular}{ | l | p{10cm} |  }
		 \hline
		Scenario name & \underline{Find the movie The Hobbit}  \\ \hline
		Participating Actors & \underline{John} \\ \hline
		Flow of Events & \begin{enumerate}
						\item John wants to find out the name of the actor playing the protagonist in The Hobbit. He enters the title in the search field.
						\item John receive information for the movie, such as a list of actors, roles, a description for the movie, rating ect.
						\end{enumerate}
						\\ \hline
						
	\end{tabular}
\end{center}


\begin{center}
	\begin{tabular}{ | l | p{10cm} |  }
		 \hline
		Scenario name & \underline{Find the actor Leonado Dicaprio}  \\ \hline
		Participating Actors & \underline{John} \\ \hline
		Flow of Events & \begin{enumerate}
						\item John liked the performance of Leonado Dicaprio in Titanic, and wants to find other movies where he stars in. He searches for Leonardo Dicaprio and recieves his personal information. Ex. Gender and a list of movies he stars in.
						\end{enumerate}
						\\ \hline
						
	\end{tabular}
\end{center}


\begin{center}
	\begin{tabular}{ | l | p{10cm} |  }
		 \hline
		Scenario name & \underline{Edit name of the actor Will Smith}  \\ \hline
		Participating Actors & \underline{John} \\ \hline
		Flow of Events & \begin{enumerate}
						\item John has discovered that Will Smith is incorrectly called Woll Smoth, and needs to change the name so it is spelled correctly. He searches for and find woll smoth.
						\item John change the name to Will Smith.
						\end{enumerate}
						\\ \hline
						
	\end{tabular}
\end{center}

\begin{center}
	\begin{tabular}{ | l | p{10cm} |  }
		 \hline
		Scenario name & \underline{Edit the movie The Hobbit}  \\ \hline
		Participating Actors & \underline{John} \\ \hline
		Flow of Events & \begin{enumerate}
						\item John notices that their is a mistake in the year of 'The Hobbit: An Unexpected Journey. The year is set to 2001.
						\item John edits the year to 2012.
						\end{enumerate}
						\\ \hline
						
	\end{tabular}
\end{center}

\begin{center}
	\begin{tabular}{ | l | p{10cm} |  }
		 \hline
		Scenario name & Find other movies that the actor who plays Bilbo in The Hobbit has starred in  \\ \hline
		Participating Actors & \underline{John} \\ \hline
		Flow of Events & \begin{enumerate}
						\item John likes the actor who plays Bilbo Baggins in 'The Hobbit: An Unexpected Journey', and want to know what other movies he starred in.
						John searches for 'The Hobbit', enters it and notices the list of actors and what they play in the movie.
						\item John clicks the name of the actor who plays Bilbo Baggins, to open the actors site
						\item John checks the list of movies that the actor has starred in and finds hot fuzz.
						\end{enumerate}
						\\ \hline
	\end{tabular}
\end{center}

\subsubsection{Use case model}

\begin{itemize}
	\setlength{\itemsep}{-5pt}
	\item User
	\item Server
\end{itemize}

Through the inspection of the scenarios depicted in the Scenario’s section, we have deduced the use cases that the program should be able to support.

The use cases are as follows:

User use cases:
\begin{itemize}
	\setlength{\itemsep}{-5pt}
	\item Find movie
	\item Find actor
	\item Find a movie through an actor
	\item Find an actor through a movie
	\item Edit movie information
	\item Edit actor information
\end{itemize}

After refining the system design, we have added the following boundary scenarios:
\begin{itemize}
	\setlength{\itemsep}{-5pt}
	
	\item StartWebServer
	\item ShutdownWebServer
	\item ConfigureWebServer
	\item ConnectionLostException
\end{itemize}

\begin{figure}[H]
\includegraphics[width=\linewidth]{img/RAD/usecasediagram.png}
\caption{Use case diagram}
\label{fig:use case diagram}
\end{figure}

To have a more thorough understanding of each individual use cases, the following explanations was produced, describing each and every step of each use case.
The explanations furthermore include the relation between the actors and the use cases


\begin{center}
	\begin{tabular}{ | l | p{10cm} |  }
		 \hline
		Use Case Name & Find movie \\ \hline
		Participating Actors & Initiated by \emph{User} \\ \hline
		Flow of Events & \begin{enumerate}
						\item[1.] The \emph{user} searches for a movie
						\begin{enumerate}
							\item[2.] The client presents the user with a list of movies with names containing the input
						\end{enumerate}
						\item[3.] The \emph{user} selects one of the search results
						\begin{enumerate}
							\item[4.] The client displays a page containing information on the movie
						\end{enumerate}
					\end{enumerate} \\ \hline
		Entry Condition & \begin{itemize}
						\item The \emph{user} has a client open
					\end{itemize} \\ \hline
		Exit Condition & \begin{itemize}
						\item The \emph{user} successfully found a movie
					\end{itemize} \\
		\hline
	\end{tabular}
\end{center}

\begin{center}
	\begin{tabular}{ | l | p{10cm} |  }
		 \hline
		Use Case Name & Find actor \\ \hline
		Participating Actors & Initiated by \emph{User} \\ \hline
		Flow of Events & \begin{enumerate}
						\item[1.] The \emph{user} searches for an actor
						\begin{enumerate}
							\item[2.] The client presents the user with a list of actors with names containing the input
						\end{enumerate}
						\item[3.] The \emph{user} selects one of the search results
						\begin{enumerate}
							\item[4.] The client displays a page containing information on the actor
						\end{enumerate}
					\end{enumerate} \\ \hline
		Entry Condition & \begin{itemize}
						\item The \emph{user} has a client open
					\end{itemize} \\ \hline
		Exit Condition & \begin{itemize}
						\item The \emph{user} successfully found an actor
					\end{itemize} \\
		\hline
	\end{tabular}
\end{center}

\begin{center}
	\begin{tabular}{ | l | p{10cm} |  }
		 \hline
		Use Case Name & Find a movie through an actor \\ \hline
		Participating Actors & Initiated by \emph{User} \\ \hline
		Flow of Events & \begin{enumerate}
						\item[1.] The \emph{user} searches for an actor
						\begin{enumerate}
							\item[2.] The client presents the user with a list of actors with names containing the input
						\end{enumerate}
						\item[3.] The \emph{user} selects one of the search results
						\begin{enumerate}
							\item[4.] The client displays a page containing information on the actor
						\end{enumerate}
						\item[5.] The \emph{user} selects one of the movies from the actor's "starred in" list
						\begin{enumerate}
							\item[6.] The client displays a page containing information on the movie
						\end{enumerate}
					\end{enumerate} \\ \hline
		Entry Condition & \begin{itemize}
						\item The \emph{user} has a client open
					\end{itemize} \\ \hline
		Exit Condition & \begin{itemize}
						\item The \emph{user} successfully found a movie
					\end{itemize} \\
		\hline
	\end{tabular}
\end{center}


\begin{center}
	\begin{tabular}{ | l | p{10cm} |  }
		 \hline
		Use Case Name & Find an actor through a actor \\ \hline
		Participating Actors & Initiated by \emph{User} \\ \hline
		Flow of Events & \begin{enumerate}
						\item[1.] The \emph{user} searches for a movie
						\begin{enumerate}
							\item[2.] The client presents the user with a list of movies with title containing the input
						\end{enumerate}
						\item[3.] The \emph{user} selects one of the search results
						\begin{enumerate}
							\item[4.] The client displays a page containing information on the movie
						\end{enumerate}
						\item[5.] The \emph{user} selects one of the actors from the movie's "starred in" list
						\begin{enumerate}
							\item[6.] The client displays a page containing information on the actor
						\end{enumerate}
					\end{enumerate} \\ \hline
		Entry Condition & \begin{itemize}
						\item The \emph{user} has a client open
					\end{itemize} \\ \hline
		Exit Condition & \begin{itemize}
						\item The \emph{user} successfully found an actor
					\end{itemize} \\
		\hline
	\end{tabular}
\end{center}


\begin{center}
	\begin{tabular}{ | l | p{10cm} |  }
		 \hline
		Use Case Name & Edit movie information \\ \hline
		Participating Actors & Initiated by \emph{administrator} \\ \hline
		Flow of Events & \begin{enumerate}
						\item[1.] The \emph{administrator} uses the client to search for, and find, a movie. On the movie page he choses to edit the movie
						\begin{enumerate}
							\item[2.] The client presents the \emph{administrator} with a page where he can edit information
						\end{enumerate}
						\item[3.] The \emph{administrator} edits the desired information, and chooses to save his changes
						\begin{enumerate}
							\item[4.] The client saves the changes
						\end{enumerate}
					\end{enumerate} \\ \hline
		Entry Condition & \begin{itemize}
						\item The \emph{administrator} has logged in and has selected a movie
					\end{itemize} \\ \hline
		Exit Condition & \begin{itemize}
						\item The \emph{administrator} has succesfully edited the movie, and the changes are saved to the database
					\end{itemize} \\
		\hline
	\end{tabular}
\end{center}




\begin{center}
	\begin{tabular}{ | l | p{10cm} |  }
		 \hline
		Use Case Name & Edit actor information \\ \hline
		Participating Actors & Initiated by \emph{administrator} \\ \hline
		Flow of Events & \begin{enumerate}
						\item[1.] The \emph{administrator} uses the client to search for, and find, an actor. On the actor page, he choses to edit the actor
						\begin{enumerate}
							\item[2.] The client presents the \emph{administrator} with a page where he can edit information on the actor
						\end{enumerate}
						\item[3.] The \emph{administrator} edits the desired information, and chooses to save his changes
						\begin{enumerate}
							\item[4.] The client saves the changes to the database
						\end{enumerate}
					\end{enumerate} \\ \hline
		Entry Condition & \begin{itemize}
						\item The \emph{administrator} has logged in and has selected an actor
					\end{itemize} \\ \hline
		Exit Condition & \begin{itemize}
						\item The \emph{administrator} has succesfully edited the actor, and the changes are saved to the database
					\end{itemize} \\
		\hline
	\end{tabular}
\end{center}



\begin{center}
	\begin{tabular}{ | l | p{10cm} |  }
		 \hline
		Use Case Name & Start Web Server \\ \hline
		Participating Actors & Initiated by \emph{administrator} \\ \hline
		Flow of Events & \begin{enumerate}
						\item[1.] The \emph{administrator} executes the server program
						\begin{enumerate}
							\item[2.] The server initializes the required classes and sets up the repository with the default settings
						\end{enumerate}
					\end{enumerate} \\ \hline
		Entry Condition & \begin{itemize}
						\item The \emph{administrator} is sitting by a computer capable of running the server application
					\end{itemize} \\ \hline
		Exit Condition & \begin{itemize}
						\item The web server has successfully been started
					\end{itemize} \\
		\hline
	\end{tabular}
\end{center}

\begin{center}
	\begin{tabular}{ | l | p{10cm} |  }
		 \hline
		Use Case Name & Shutdown Web Server \\ \hline
		Participating Actors & Initiated by \emph{administrator} \\ \hline
		Flow of Events & \begin{enumerate}
						\item[1.] The \emph{administrator} closes the server program
						\begin{enumerate}
							\item[2.] The server stops receiving requests.
							\item[3.] The server finishes any current transaction request.
							\item[4.] The server closes the application.			 
						\end{enumerate}
					\end{enumerate} \\ \hline
		Entry Condition & \begin{itemize}
						\item The \emph{administrator} is sitting by a computer running the server application
					\end{itemize} \\ \hline
		Exit Condition & \begin{itemize}
						\item The server has successfully shut down
					\end{itemize} \\
		\hline
	\end{tabular}
\end{center}

\begin{center}
	\begin{tabular}{ | l | p{10cm} |  }
		 \hline
		Use Case Name & Server Crash Exception \\ \hline
		Participating Actors & Initiated by \emph{server} \\ \hline
		Flow of Events & \begin{enumerate}
						\item[1.] The \emph{server} program crashes
						\item[2.] The database system rollbacks any incomplete transactions.
						\end{enumerate} \\ \hline
		Entry Condition & \begin{itemize}
						\item The \emph{server} program stopped responding
					\end{itemize} \\ \hline
		Exit Condition & \begin{itemize}
						\item The server program is closed and any errors has been handled
					\end{itemize} \\
		\hline
	\end{tabular}
\end{center}

\begin{center}
	\begin{tabular}{ | l | p{10cm} |  }
		 \hline
		Use Case Name & Connection Lost Exception \\ \hline
		Participating Actors & Initiated by \emph{server} \\ \hline
		Flow of Events & \begin{enumerate}
						\item[1.] The \emph{server} completes any current transaction
						\item[2.] The server logs all relevant information for each transaction to be used when the server regains connection
						\item[3.] The server waits until it regains connection
						\end{enumerate} \\ \hline
		Entry Condition & \begin{itemize}
						\item The \emph{server} program lost connection
					\end{itemize} \\ \hline
		Exit Condition & \begin{itemize}
						\item The server program has handled every current transaction and logged them
					\end{itemize} \\
		\hline
	\end{tabular}
\end{center}

\subsubsection{Object model}

\paragraph{Entity objects}

\begin{enumerate}
	\item[1.] Movie \hfill \\
	A movie is a collection of data about a specific movie. It has attributes defining its title, length, year of production and genre. Some movies are episodes of series and has attributes defining it's season and episode number.
	
	\item[2.] Person \hfill \\
	The person entity defines data about a specific person. Each entity instance includes attributes defining the name and the gender of the person. The person links to zero or more person information, describing further information such as age.
	
\end{enumerate}

\paragraph{Boundary objects}

\begin{enumerate}

	\item Client \hfill \\
	The general object for the actual client that is being used. This boundary object can represent both the Web Client and the Desktop Client

	\item SearchField \hfill \\
	The main search field where you can search for searchable content like movies and actors.
	
	\item SearchResults \hfill \\
	The object defining the form that is presented when the user has receive a list of resulsts from a search. The object is used when the user interacts with the result list.
	
	
	\item MovieInformationForm \hfill \\
	A form where a user can input data about a movie
	
	\item ActorInformationForm \hfill \\
	A form where a user can input data about an Actor
	
	
\end{enumerate}

\paragraph{Control objects}

These controllers are all more or less self-explanatory. They enable the user to realise all use cases from their client.

\begin{enumerate}
	\item[1.] FindMovieController
	\item[2.] FindActorController
 	\item[10.] EditMovieController
 	\item[13.] EditActorController
 	
\end{enumerate}

\paragraph{Class Diagram}

\begin{figure}[H]
\includegraphics[width=\linewidth]{img/RAD/ClassDiagram.png}
\caption{Class diagram}
\label{fig:Class diagram}
\end{figure}

\subsubsection{Dynamic model}

\paragraph{Sequence diagrams}

The sequence diagrams explains some of the more apparant use cases of the system in detail. The interactions shown in the diagrams are not directly linked to the final interaction of the actual program, but is describing the overall flow of the messages sent between the objects being used.

\begin{figure}[H]
\includegraphics[width=\linewidth]{img/RAD/SearchSequenceDiagram.png}
\caption{Search Sequence Diagram}
\label{fig:Search Sequence Diagram}
\end{figure}

\begin{figure}[H]
\includegraphics[width=\linewidth]{img/RAD/EditMovieSequenceDiagram.png}
\caption{Edit Movie Sequence Diagram}
\label{fig:Edit Movie Sequence Diagram}
\end{figure}

\paragraph{State machines}
An entity can have different states. As seen in figure \ref{fig:Entity State Machine Diagram}, after its creation (POST), it is stored in the persistent database module on the server. From here it can either be deleted (DELETE), or it can be fetched (GET). When a object is queried from the database the server will fetch it as a DTO (Data transfer object) into the servers memory and the DTO is sent to the client requesting the object. A DTO can from there either be updated (PUT) and pushed back to the server or it can go out of scope, meaning it's removed from the server's memory. (This means that only the DTO will be deleted and not the entity in the database).

This is valid for Movies and Actors.

\begin{figure}[H]
\includegraphics[width=\linewidth]{img/RAD/EntityStateMachineDiagram.png}
\caption{Entity State Machine Diagram}
\label{fig:Entity State Machine Diagram}
\end{figure}
\newpage
