\chapter{Testing Strategy}
	
	While the definition of done stated that code should be tested, there was no rule as to in what manner. The individual developer was in charge of their own code, determining whether to use test-driven development or code first.
	To maintain high quality of the software being committed during the process of making the program, each developer was tasked to perform tests that thoroughly inspected each possible outcome of their methods in each individual class.
	
	There were no further rules about the structure of these tests, meaning that the individual developer themselves chose whether or not to utilize the array of testing methodologies \footnote{equivalence, boundary, state-base, path and polymorphism testing} during the testing phase.
	
	Most tests have been designed using the testing methodologies. To increase production, and lessen the time spent on testing, the majority of tests combine the different approaches, while maintaining the quality factor of each testing formality.

	Integration tests were conducted on groups of subsystems, to test the combined functionality. The quality and thoroughness of the integration tests were limited to only test the exact interactions between the subsystems, since any other fault should be caught during unit testing.
	
	To ensure the quality of tests, each test has been reviewed by another developer, tasked to inspect the validity and thoroughness of the tests being conducted.